
\subsection{Register interface}
This interface uses a TCP/IP server based on the C++ library code described above. It maps
the functionality of the library onto a simple register-based interface.

Communication is connection-based, via TCP sockets.

\subsubsection{Basic protocol}
The protocol is similar to the firmware protocol described in
section~\ref{protoc}, except that the ``destination'' object --- the object
whose registers we wish to inspect or change --- is now a wheel, rather than a slave controller.

As in that protocol, client to server messages are of variable length, and are
preceded by the message length in bytes and a single byte encoding both the message
type and the desired wheel (message type in bottom 4 bits, wheel in top 4); while server
responses are either a single byte response or a set of 
register values. The message types are write, set a read set, and read (to which
the server responds by sending the values of the registers specified in the given read set.) For more
details, see the section referenced above; but it's worth mentioning here that the
\textbf{read sets are shared across all devices.}
Some examples:
\subsubsection{Example 1: write}
PC to master: \verb+0d 32 04 00 ff 01 ff 02 01 00 03 02 00+ \\
Response: \verb+00+ 
\begin{itemize}
\item \verb+0d 32+:  13 bytes, wheel 3, command 2 (write)
\item \verb+04+: contains 4 writes
\item \verb+00 ff+ write \verb+ff+ to register 0
\item \verb+01 ff+ write \verb+ff+ to register 1
\item \verb+02 01 00 + write \verb+0001+ to register 2
\item \verb+03 02 00 + write \verb+0002+ to register 3
\end{itemize}
Response is:
\begin{itemize}
\item 0 : success
\item 1 : register out of range (i.e. that register does not exist)
\item 2 : register is not writable
\item 3 : register write not yet implemented for this register
\end{itemize}


\subsubsection{Example 2: set read set}
PC to master: \verb+07 04 01 00 01 02 03+ \\
Response: \verb+04+ 
\begin{itemize}
\item \verb+07 24+: 7 bytes, wheel 0, command 4 (set read set.) We specify wheel 0 because read sets 
are shared across all devices --- the wheel number is ignored.
\item \verb+01+ : read set 1
\item \verb+00 01 02 03+ : registers 0, 1, 2 and 3
should be the new read set 1.
\end{itemize}
The response is \verb+04+, the number of registers in the read set.

\subsubsection{Example 3: read}
PC to master: \verb+03 12 01+ \\
Response: \verb+ff ff 01 00 02 00+ 
\begin{itemize}
\item \verb+03 12+ : 3 bytes, wheel 1, command 2 (read)
\item \verb+01+ : return the values of the registers in read set 1
\end{itemize}
The response is the values of the registers:
\begin{itemize}
\item \verb+ff+ : register 0 contains \verb+ff+ 
\item \verb+ff+ : register 1 contains \verb+ff+ 
\item \verb+01 00+ : register 2 contains \verb+0001+ 
\item \verb+02 00+ : register 3 contains \verb+0002+ 
\end{itemize}

\subsubsection{Wheel Registers}
These registers are \textbf{per wheel}: the wheel number must be nonzero.

\vspace*{1em}

\begin{tabular}{|p{0.2in}|p{2.7in}|p{0.1in}|p{0.1in}|p{1in}|p{1.5in}|}\hline
\textbf{ID} & \textbf{Symbol} & \textbf{n} & \textbf{W} & \textbf{Mapping} & \textbf{Description}  \\ \hline 
0 & REGW\_DRIVE\_RESET & 1 & Y & unmapped & set nonzero to reset overcurrent\\ \hline
1 & REGW\_DRIVE\_REQSPEED & 2 & Y & [-4000.0,4000.0] & required speed\\ \hline
2 & REGW\_DRIVE\_PGAIN & 2 & Y & [0.0,10.0] & P-gain\\ \hline
3 & REGW\_DRIVE\_IGAIN & 2 & Y & [0.0,10.0] & I-gain\\ \hline
4 & REGW\_DRIVE\_DGAIN & 2 & Y & [-10.0,10.0] & D-gain\\ \hline
5 & REGW\_DRIVE\_INTEGRALCAP & 2 & Y & [0.0,1000.0] & integral error cap\\ \hline
6 & REGW\_DRIVE\_INTEGRALDECAY & 2 & Y & [0.0,1.0] & integral decay\\ \hline
7 & REGW\_DRIVE\_OVERCURRENTTHRESH & 2 & Y & [0.0,1000.0] & overcurrent threshold\\ \hline
8 & REGW\_DRIVE\_ACTUALSPEED & 2 &  & [-4000.0,4000.0] & actual speed from encoder\\ \hline
9 & REGW\_DRIVE\_ERROR & 2 &  & [-1000.0,1000.0] & required minus actual speed\\ \hline
10 & REGW\_DRIVE\_ERRORINTEGRAL & 2 &  & [-1000.0,1000.0] & error integral magnitude\\ \hline
11 & REGW\_DRIVE\_ERRORDERIV & 2 &  & [-200.0,200.0] & error derivative\\ \hline
12 & REGW\_DRIVE\_CONTROL & 2 &  & [-1024.0,1024.0] & value being sent to motor\\ \hline
13 & REGW\_DRIVE\_INTERVALCTRL & 2 &  & [0.0,1000.0] & time between control runs (ms)\\ \hline
14 & REGW\_DRIVE\_CURRENT & 2 &  & unmapped & raw current reading\\ \hline
15 & REGW\_STEER\_RESET & 1 & Y & unmapped & set nonzero to reset overcurrent\\ \hline
16 & REGW\_STEER\_REQPOS & 2 & Y & [0.0,1024.0] & required position\\ \hline
17 & REGW\_STEER\_PGAIN & 2 & Y & [0.0,10.0] & P-gain\\ \hline
18 & REGW\_STEER\_IGAIN & 2 & Y & [0.0,10.0] & I-gain\\ \hline
19 & REGW\_STEER\_DGAIN & 2 & Y & [-10.0,10.0] & D-gain\\ \hline
20 & REGW\_STEER\_INTEGRALCAP & 2 & Y & [0.0,1000.0] & integral error cap\\ \hline
\end{tabular}

\begin{tabular}{|p{0.2in}|p{2.7in}|p{0.1in}|p{0.1in}|p{1in}|p{1.5in}|}\hline
\textbf{ID} & \textbf{Symbol} & \textbf{n} & \textbf{W} & \textbf{Mapping} & \textbf{Description}  \\ \hline 
21 & REGW\_STEER\_INTEGRALDECAY & 2 & Y & [0.0,1.0] & integral decay\\ \hline
22 & REGW\_STEER\_OVERCURRENTTHRESH & 2 & Y & [0.0,1000.0] & overcurrent threshold\\ \hline
23 & REGW\_STEER\_ACTUALPOS & 2 &  & [0.0,1024.0] & actual position from pot\\ \hline
24 & REGW\_STEER\_ERROR & 2 &  & [-1000.0,1000.0] & required minus actual position\\ \hline
25 & REGW\_STEER\_ERRORINTEGRAL & 2 &  & [-1000.0,1000.0] & error integral magnitude\\ \hline
26 & REGW\_STEER\_ERRORDERIV & 2 &  & [-200.0,200.0] & error derivative\\ \hline
27 & REGW\_STEER\_CONTROL & 2 &  & [-1024.0,1024.0] & value being sent to motor\\ \hline
28 & REGW\_STEER\_INTERVALCTRL & 2 &  & [0.0,1000.0] & time between control runs (ms)\\ \hline
29 & REGW\_STEER\_CURRENT & 2 &  & unmapped & raw current reading\\ \hline
30 & REGW\_STEER\_CALIBMIN & 2 &  & [-512.0,512.0] & calibration minimum\\ \hline
31 & REGW\_STEER\_CALIBMAX & 2 &  & [-512.0,512.0] & calibration minimum\\ \hline
32 & REGW\_LIFT\_RESET & 1 & Y & unmapped & set nonzero to reset overcurrent\\ \hline
33 & REGW\_LIFT\_REQPOS & 2 & Y & [0.0,1024.0] & required position\\ \hline
34 & REGW\_LIFT\_PGAIN & 2 & Y & [0.0,10.0] & P-gain\\ \hline
35 & REGW\_LIFT\_IGAIN & 2 & Y & [0.0,10.0] & I-gain\\ \hline
36 & REGW\_LIFT\_DGAIN & 2 & Y & [-10.0,10.0] & D-gain\\ \hline
37 & REGW\_LIFT\_INTEGRALCAP & 2 & Y & [0.0,1000.0] & integral error cap\\ \hline
38 & REGW\_LIFT\_INTEGRALDECAY & 2 & Y & [0.0,1.0] & integral decay\\ \hline
39 & REGW\_LIFT\_OVERCURRENTTHRESH & 2 & Y & [0.0,1000.0] & overcurrent threshold\\ \hline
40 & REGW\_LIFT\_ACTUALPOS & 2 &  & [0.0,1024.0] & actual position from pot\\ \hline
41 & REGW\_LIFT\_ERROR & 2 &  & [-1000.0,1000.0] & required minus actual position\\ \hline
42 & REGW\_LIFT\_ERRORINTEGRAL & 2 &  & [-1000.0,1000.0] & error integral magnitude\\ \hline
43 & REGW\_LIFT\_ERRORDERIV & 2 &  & [-200.0,200.0] & error derivative\\ \hline
44 & REGW\_LIFT\_CONTROL & 2 &  & [-1024.0,1024.0] & value being sent to motor\\ \hline
45 & REGW\_LIFT\_INTERVALCTRL & 2 &  & [0.0,1000.0] & time between control runs (ms)\\ \hline
46 & REGW\_LIFT\_CURRENT & 2 &  & unmapped & raw current reading\\ \hline
47 & REGW\_LIFT\_CALIBMIN & 2 &  & [-512.0,512.0] & calibration minimum\\ \hline
48 & REGW\_LIFT\_CALIBMAX & 2 &  & [-512.0,512.0] & calibration minimum\\ \hline
\end{tabular}
\clearpage
\subsubsection{Master registers}
The registers below are master registers for the whole system, for which the wheel number
should be zero. Note that the exceptions are handled by individual registers for each slave device rather than
by wheels --- this is because there is no 1-to-1 mapping between wheels and devices.

\vspace*{1em}

\begin{tabular}{|p{0.2in}|p{2.3in}|p{0.1in}|p{0.1in}|p{0.7in}|p{2in}|}\hline
\textbf{ID} & \textbf{Symbol} & \textbf{n} & \textbf{W} & \textbf{Mapping} & \textbf{Description}  \\ \hline 
0 & REGM\_TEMP1 & 2 &  & [0.0,1024.0] & temperature sensor\\ \hline
1 & REGM\_TEMP2 & 2 &  & [0.0,1024.0] & temperature sensor\\ \hline
2 & REGM\_TEMP3 & 2 &  & [0.0,1024.0] & temperature sensor\\ \hline
3 & REGM\_TEMP4 & 2 &  & [0.0,1024.0] & temperature sensor\\ \hline
4 & REGM\_TEMP5 & 2 &  & [0.0,1024.0] & temperature sensor\\ \hline
5 & REGM\_TEMP6 & 2 &  & [0.0,1024.0] & temperature sensor\\ \hline
6 & REGM\_RESET & 1 &  & unmapped & set to stop and reset system\\ \hline
7 & REGM\_DEBUGLED & 1 & Y & unmapped & debugging LEDs, turns on for some time\\ \hline
8 & REGM\_TIMER & 2 &  & unmapped & millis since start\\ \hline
9 & REGM\_INTERVALI2C & 2 &  & unmapped & interval between I2C ticks\\ \hline
10 & REGM\_STATUS & 2 &  & unmapped & status flags\\ \hline
11 & REGM\_EXCEPT1 & 1 &  & unmapped & exception state of device 1\\ \hline
12 & REGM\_EXCEPT2 & 1 &  & unmapped & exception state of device 2\\ \hline
13 & REGM\_EXCEPT3 & 1 &  & unmapped & exception state of device 3\\ \hline
14 & REGM\_EXCEPT4 & 1 &  & unmapped & exception state of device 4\\ \hline
15 & REGM\_EXCEPT5 & 1 &  & unmapped & exception state of device 5\\ \hline
16 & REGM\_EXCEPT6 & 1 &  & unmapped & exception state of device 6\\ \hline
17 & REGM\_EXCEPT7 & 1 &  & unmapped & exception state of device 7\\ \hline
18 & REGM\_EXCEPT8 & 1 &  & unmapped & exception state of device 8\\ \hline
19 & REGM\_EXCEPT9 & 1 &  & unmapped & exception state of device 9\\ \hline
20 & REGM\_DISABLEDEXCEPTIONS1 & 1 & Y & unmapped & bitmap of disabled exceptions in dev. 1\\ \hline
21 & REGM\_DISABLEDEXCEPTIONS2 & 1 & Y & unmapped & bitmap of disabled exceptions in dev. 2\\ \hline
22 & REGM\_DISABLEDEXCEPTIONS3 & 1 & Y & unmapped & bitmap of disabled exceptions in dev. 3\\ \hline
23 & REGM\_DISABLEDEXCEPTIONS4 & 1 & Y & unmapped & bitmap of disabled exceptions in dev. 4\\ \hline
24 & REGM\_DISABLEDEXCEPTIONS5 & 1 & Y & unmapped & bitmap of disabled exceptions in dev. 5\\ \hline
25 & REGM\_DISABLEDEXCEPTIONS6 & 1 & Y & unmapped & bitmap of disabled exceptions in dev. 6\\ \hline
26 & REGM\_DISABLEDEXCEPTIONS7 & 1 & Y & unmapped & bitmap of disabled exceptions in dev. 7\\ \hline
27 & REGM\_DISABLEDEXCEPTIONS8 & 1 & Y & unmapped & bitmap of disabled exceptions in dev. 8\\ \hline
28 & REGM\_DISABLEDEXCEPTIONS9 & 1 & Y & unmapped & bitmap of disabled exceptions in dev. 9\\ \hline
29 & REGM\_TIMEOUT & 1 & Y & [0.0,30.0] & server timeout in seconds\\ \hline
\end{tabular}

\subsubsection{Multiple clients}
The server can handle up to 4 clients connected at any one time (and this
number can easily be changed in the code.) All state is shared across
the clients --- notably, changing the read set on one client will change it
on all of them, because they all communicate with the same devices!

\subsubsection{Disconnection}
The server will detect if a client closes its end of the socket, and will
disconnect itself, freeing up the client slot. However, this will not
deal with all circumstances. To this end, there is a timeout register
\verb+REGM_TIMEOUT+. If a client does not communicate with the server
within this time, it is considered to be disconnected by the server,
and the server end of the socket is closed. The client should
do something similar. The default value is 5 seconds.
