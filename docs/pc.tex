%%%%%%%%%%%%%%%%%%%%%%%%%%%%%%%%%%%%%%%%%%%%%%%%%%%%%%%%%%%%%%%%%%%%%%%%%%%%%
%
%  System        : 
%  Module        : 
%  Object Name   : $RCSfile$
%  Revision      : $Revision$
%  Date          : $Date$
%  Author        : $Author$
%  Created By    : Jim Finnis
%  Created       : Mon Dec 10 22:01:02 2012
%  Last Modified : <250213.1532>
%
%  Description 
%
%  Notes
%
%  History
% 
%%%%%%%%%%%%%%%%%%%%%%%%%%%%%%%%%%%%%%%%%%%%%%%%%%%%%%%%%%%%%%%%%%%%%%%%%%%%%
%
% Copyright (c) 2012 Jim Finnis.
% 
% All Rights Reserved.
% 
% This  document  may  not, in  whole  or in  part, be  copied,  photocopied,
% reproduced,  translated,  or  reduced to any  electronic  medium or machine
% readable form without prior written consent from Jim Finnis.
%
%%%%%%%%%%%%%%%%%%%%%%%%%%%%%%%%%%%%%%%%%%%%%%%%%%%%%%%%%%%%%%%%%%%%%%%%%%%%%

\section{Onboard PC library}
The onboard PC currently uses a library of C++ classes to abstract
the communications protocol, the details of \isqc{}, and
the vagaries of the firmware register set (for the reasons given
in section~\ref{firmrec}, working out which slave controller to send
commands to in order to turn a particular wheel is not straightforward.)

This section contains a brief overview of the C++ classes, describing their
functions but not going into any details of methods.


\subsection{Rover}
The \emph{Rover} class is the top-level class. The utilising code is expected
to instantiate a single object of this class. Calling the initialisation method
\emph{init} 
with a character device name and a baud rate will start a connection to the master
controller (the Arduino) which will be reset. Once the connection is made,
calling \emph{update} periodically will request state information about all the 
devices.

Other methods allow the user access to motor objects, which will allow the user
to change motor parameters and required speeds/positions; and to motor data
blocks, allowing users to monitor the motors. The data blocks will contain
information received at the last \emph{update.} 

Note that the communications over \isqc{} have been kept to a minimum --- the
rover uses the ``set read set'' firmware protocol command (see
section~\ref{protoc}) to send a set of registers. When a simple 3 byte ``read
set'' command is set with a set number, the slave will send a packet of all
the registers in that set.

Internally, the rover uses the \emph{WheelPair} class to manage triplets
of slave controller boards, which in turn use \emph{MotorDriverData} objects
to manage reading data from the two types of slave (the \emph{Motor} class
handles writing values.) 

Communications are handled by \emph{SlaveProtocol}, which in turn uses
\emph{SerialComms.} 

\subsubsection{Creating and initialising the Rover object}
You should obtain a pointer to the Rover singleton
by calling \texttt{getInstance} on the class:
\begin{v}
    Rover *r = Rover::getInstance();
\end{v}
This ensures there can only ever be one Rover object. You can then 
initialise the connection and send the default calibration data:
\begin{v}
    // provide init() with the serial port to be used (an optional
    // second argument gives data about which wheels actually exist).
    
    r->init("/dev/ttyACM0");
    r->calibrate();
\end{v}


\subsection{Motor}
This class contains methods to send parameters (such as PID gains) to a motor,
set its required position or speed, and reset any exception condition\footnote{for example, if
a motor's current exceeds its overcurrent threshold, the slave will detect this
and shut the motor down.}. There are three subclasses of \emph{Motor}, one
for each motor type: \emph{DriveMotor}, \emph{SteerMotor} and \emph{LiftMotor.} They present similar interfaces to the user.
Pointers to motor objects can be obtained by calling methods in \emph{Rover}.

\subsection{MotorData}
This is a class containing general motor monitoring data, such as:
\begin{itemize}
\item error, error integral, error derivative
\item control value being sent to the motor
\item interval between control ticks
\item current
\item actual position or speed
\item exception state
\end{itemize}
There are three subclasses, one for each motor type: the lift and steer
subclasses are identical and the drive subclass also contains odometry data.

Pointers to motor data objects can be obtained by calling methods in \emph{Rover}.
Note that the data in these objects is \textbf{only updated when \emph{update} is called.} 


\subsection{StatusListener}
The communications system can inform third parties who register with it of changes
to the comms status --- for example, disconnection or protocol error. This is done
by creating a \emph{StatusListener} subclass and registering it with the \emph{SerialComms} 
object, available through \emph{Rover.} 

\subsection{Calibration settings}
Default calibration settings for the motors can (and should) be sent
to the controllers by calling
\texttt{Rover::calibrate()} in the file \texttt{rover.cpp}.



\subsection{Example code}
\begin{v}
int main(int argc,char *argv[]){
    Rover *r = Rover::getInstance();
    
    try {
        // set up the rover given the comms port and the
        // baud rate and the wheel pair mask (default is 7,
        // all wheel pairs present)
        r->init("/dev/ttyACM0");
        
        // send default calibration
        r->calibrate();
        
        // some parameter data we're going to change;
        // you probably won't do this - it's just an illustration.

        MotorParams params = {
            0.004,0,0,   //PID
            0,0,         //integral cap and decay
            300,         //overcurrent threshold
        };
        
        // change parameters on the drive motors
        // and set a speed for them
        
        for(int i=1;i<=6;i++) { // motors are 1 to 6 as in the documentation
            Motor *m = r->getDrive(i); // get each drive motor

            // get a pointer to its parameters
            MotorParams *p = m->getParams();

            // copy some other data into them
            *p = params;

            // and send the changes
            m->sendParams();

            // and set a speed
            m->setRequired(1000);
        }
        
        for(;;){
            usleep(10000); // wait 1/100 s
            r->update(); // update the rover

            // get drive motor 1 data
            DriveMotorData *d = r->getDriveData(1);
            printf("%f\n",d->actual); // print actual speed
        }   
        
    } catch(SlaveException e) {

        // slave exceptions are thrown by protocol and comms errors
        printf("Error in rover communication: %s\n",e.msg);
        return 0;
    }
}        
    

\end{v}


